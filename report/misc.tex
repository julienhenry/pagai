
\documentclass[a4paper,english,11pt]{article}

\usepackage{mathcomp,amssymb}
\usepackage[utf8]{inputenc}
\usepackage[T1]{fontenc}
\usepackage{amsmath}
\usepackage{mathenv}
\usepackage{tikz}
\usepackage{listings}
\usepackage{placeins}
\usepackage{algorithm}
\usepackage{algpseudocode}
\usepackage{multirow}
\usepackage[toc,page]{appendix} 
% DM: verimagstudent charge hyperref, et hyperref doit habituellement
% DM: passer en dernier
\newtheorem{definition}{Definition}[section]
\newtheorem{theorem}{Theorem}[section]
\newtheorem{proposition}[theorem]{Proposition}

\newcommand*\system[1]{\left[ \begin{array}{lllll}#1 \end{array}\right.}


\def\R{\mathbb{R}}
\def\Z{\mathbb{Z}}
\def\N{\mathbb{N}}
\def\Q{\mathbb{Q}}
\def\P{\mathcal{P}}
\def\ite{\textnormal{ite\ }}
\def\lfp{\textnormal{\it lfp}}
\newcommand{\widening}{\mathop{\triangledown}}

\definecolor{trefle}{rgb}{0,0.5,0}
\definecolor{moka}{rgb}{0.5,0.25,0}
\definecolor{minuit}{rgb}{0,0,0.5}

\tikzstyle{arrow}=[->,line width=.05cm,draw=red!90!blue!60!black]

\usetikzlibrary{snakes,arrows,shapes,backgrounds,shadows,automata,patterns}
\usepgflibrary{snakes}

\tikzstyle{state}=[rectangle,draw=black,minimum size=25pt,inner sep=0pt]
\tikzstyle{transition}=[rectangle,semithick,draw=black!75,
  			  minimum size=4mm]
\tikzstyle{transition2}=[transition,rectangle,thick,dashed,
  			  minimum size=4mm]
\tikzstyle{PRstate}=[circle,double,draw,fill=blue!15,minimum size=13pt,inner sep=0pt]
\tikzstyle{polyhedra}=[blue!25,opacity=0.5,pattern=north west lines,pattern
color=blue]
\tikzstyle{line}=[black,thick]

\lstnewenvironment{LLVM}
{\lstset{language=C,
		basicstyle=\ttfamily\footnotesize,
		commentstyle=\color{moka}\textit,
		keywordstyle=\color{minuit},
		identifierstyle=\color{trefle},
		showstringspaces=false}}
{}

\lstnewenvironment{C}
{\lstset{language=C,
		basicstyle=\ttfamily,
		commentstyle=\color{moka}\textit,
		keywordstyle=\color{minuit},
		identifierstyle=\color{trefle},
		showstringspaces=false}}
{}

\title{Path Focusing vs. Lookahead Widening}
\author{Julien Henry}


\begin{document}
\maketitle

\section{Exemple}

\begin{figure}[!h]
\centering
\begin{tikzpicture}[->,>=stealth',auto,node distance=2.5cm,
                    semithick,font=\footnotesize]

	\node[state] (n0) {$p_0$};
	\node[state] (n1) [below of=n0] {\begin{tabular}{l}
	$p_1$: \\
	$i \gets \Phi(0,i')$
	\end{tabular}};
	\node[state] (n2) [below left of=n1] {$p_2$};
	\node[state] (n3) [below right of=n2] {$p_3$};

  \path [transition] 
		(n0) edge              node {} (n1);
  \path [transition] 
        (n1) edge			   node [left] {$i \leq 50$} (n2);
  \path [transition] 
        (n2) edge	[loop left, distance=2cm]	   node  {} (n2);
  \path [transition] 
        (n2) edge			   node  {} (n3);
  \path [transition] 
        (n3) edge			   node [right] {$i' \gets i+1$} (n1);
  %\path [transition] 
  %      (n2) edge [out=0, in=0, distance=2cm] node [right] {$x = x+1$} (n1);

\end{tikzpicture}
\end{figure}

\emph{Path Focusing}:

$p_1$ et $p_2$ sont des têtes de boucles. On a $P_R = \{p_1, p_2\}$.
\begin{itemize}
\item On démarre avec $p_1$ : $i=0$. On trouve le chemin $p_1 \rightarrow p_2$.
\item $p_2$ : $i=0$. On trouve le chemin $p_2 \rightarrow p_3 \rightarrow p_1$.
On obtient en $p_1$ : $i \in [0,1]$. C'est une tête de boucle, donc on élargit
et on trouve $i \geq 0$.
\item On trouve alors le chemin $p_1 \rightarrow p_2$. l'image du polyhèdre $i
\geq 0$ par ce chemin est $0 \leq i \leq 50$. $p_2$ est également une tête de
boucle, donc on élargit\dots et on trouve $i \geq 0$.
\end{itemize}

\emph{Lookahead Widening} trouve quant à lui $0 \leq i \leq 50$.
\end{document}
