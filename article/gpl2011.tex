\documentclass{llncs}
\usepackage{a4wide}
\usepackage[french]{babel}
\usepackage[utf8]{inputenc}
\usepackage{listings}
\usepackage{placeins}
\usepackage{pgf}
\usepackage{tikz}
\usetikzlibrary{arrows,automata}

\title{Analyse statique par découverte de chemin}
\titlerunning{}
\subtitle{}
\author{Julien Henry}
\institute{CNRS - Verimag \\ \email{Julien.Henry@imag.fr}}

\begin{document}
\maketitle

\section{Introduction}

L'analyse statique consiste à trouver à la compilation des propriétés sur un
programme, comme l'ensemble des valeurs possibles pour chacunes des
variables durant l'exécution. Ceci permet par exemple de prouver que le
programme ne va pas avoir de débordements arithmétiques, à calculer des
invariants de boucle, à faire des optimisations de code lors de la compilation, etc.

L'Interprétation Abstraite est une méthode classique d'analyse statique, qui consiste à calculer une surapproximation de
l'ensemble des états possibles d'un programme numérique. 
L'état d'un programme est caractérisé par la valeur de chacunes de ses variables
numériques, et par la position à laquelle on se trouve dans le code.
Un cas particulier
d'Interprétation Abstraite est l'analyse par relations linéaires: On surapproxime
l'ensemble des états possibles du programme par un polyhèdre convexe, pour découvrir des propriétés linéaires entre les différentes variables
numériques du programme.
Cette méthode implique des surapproximations, nécessaires pour rendre
l'analyse moins coûteuse : il faut alors essayer de
trouver le bon compromis entre la précision et le coût. De nombreux travaux
consistent à trouver de nouvelles techniques pour améliorer cette précision à
moindre coût. La méthode proposée ici consiste à s'aider des
progrès dans la recherche sur le SMT-solving
pour raffiner la méthode
d'interprétation abstraite.

\section{Interprétation abstraite}

L'application de l'Interprétation Abstraite à l'analyse de
relations linéaires consiste à attacher à chaque point de contrôle du programme un polyhèdre convexe dont les dimensions sont les variables numériques. On travaille sur
le graphe de flot de contrôle du programme (voir figure) : le programme peut
être vu comme un graphe où chaque transition est étiquetée par une ou plusieurs
instructions, et peut être soumise à condition(s). On calcule itérativement les
différents polyhèdres $A_i$ à chaque noeud $s_i$ du graphe de la façon suivante:

Au départ, le polyhèdre de l'état initial est l'espace entier (aucune
contrainte sur les variables), et celui des autres états est le polyhèdre vide
(l'état est considéré comme étant inaccessible). 

Puis, on itère ce calcul pour tous les états du graphe, tant que la suite des
polyhèdres $A_i$ est croissante pour la relation d'inclusion :
\begin{itemize}
\item On part d'un état $i$, et on met à jour chaque état $j$ où $j$ est un
successeur de $i$ dans le graphe, en calculant l'image $Y$ du polyhèdre $A_i$
par la transition $i \rightarrow j$. 
\item Le nouveau polyhèdre $A_j$ est
l'enveloppe convexe de $Y$ avec l'ancienne valeur de $A_j$, notée $Y \sqcup
A_j$, qui est le plus petit polyhèdre convexe incluant $A_j$ et $Y$. Les valeurs
successives de $A_j$ au cours de l'analyse définissent donc bien une suite croissante.
\end{itemize}
Après un certain nombre d'itérations, chaque polyhèdre a atteint une limite, et
l'analyse s'arrête \footnote{En fait, pour garantir la terminaison, on a recours
à un opérateur dit d'élargissement, qui fait une extrapolation lorsqu'un
polyhèdre risque de croître indéfiniment.}. Les polyhèdres résultats permettent de déduire des
invariants sur les variables du programme.

\begin{figure}[!h]
\begin{minipage}[b]{0.5\linewidth}
\begin{lstlisting}
// position s1
x = 0;
i = 0;
//position s2
while (i < n) {
	// position s3
	i++;
	x = x+2;
}
// position s4
\end{lstlisting}
\end{minipage}
\begin{minipage}[b]{0.5\linewidth}
\begin{tikzpicture}[->,>=stealth',shorten >=1pt,auto,node distance=1.5cm,
semithick]
  \tikzstyle{every state}=[]

	\node[initial,state] (A)                    {$s_1$};
	\node[state]         (B) [below of=A] {$s_2$};
	\node[state]         (C) [below of=B] {$s_3$};
	\node[state]         (D) [left of=B]       {$s_4$};

	\path (A) edge               node {$x=0;\ i=0;$} (B);
	\path (B) edge               node {$i < n$} (C);
	\path (B) edge               node {$i\geq n$} (D);
	\path (C) edge [out=0, in=0, distance=2cm] node [right] {$i++;\ x=x+2;$} (B);


\end{tikzpicture}
\end{minipage}
\caption{Exemple de programme, et son graphe de flot de contrôle associé}
\end{figure}
%\FloatBarrier

Dans l'exemple de la figure 1, en supposant que $n>0$, l'analyse donne:
\begin{itemize}
\item $A_1$ est le polyhèdre $\{n>0\}$: au point $s_1$, on sait
seulement que $n>0$.
\item $A_2$ est le polyhèdre $\{0 \leq x \leq 2n,\ 0 \leq i \leq n\}$.
\item $A_3$ est le polyhèdre $\{0 \leq x < 2n,\ 0 \leq i < n\}$.
\item $A_4$ est le polyhèdre $\{i=n,\ x=2n\}$. On connaît donc directement la
valeur des variables à la fin du programme, ce qui permet des optimisations de
compilation.
\end{itemize}

\section{Découverte de chemin}

Nous proposons un raffinement de la méthode d'analyse par Interprétation
Abstraite, en guidant plus astucieusement les itérations de point fixe. En
effet, la vitesse et la précision de l'analyse dépend de l'ordre dans lequel on
traite les noeuds du graphe.

Au lieu de travailler sur le graphe de flot de contrôle et de calculer un
polyhèdre à chacun des noeuds, on choisit un sous-ensemble $P$ des noeuds et on fait
le calcul de point fixe sur le graphe ne contenant que les noeuds de $P$, et où
une transition correspond à un chemin possible dans le graphe de départ.

\begin{figure}
\begin{minipage}[b]{0.5\linewidth}
\begin{center}
\begin{tikzpicture}[->,>=stealth',shorten >=1pt,auto,node distance=1cm,
semithick]
  \tikzstyle{every state}=[minimum size=0.4cm]
  \tikzstyle{PRstate}=[fill=black]

	\node[initial,state,PRstate] (A)                    {};
	\node[state,PRstate]         (B) [below of=A] {};
	\node[state,PRstate]         (C) [left of=B] {};
	\node[state]         (D) [below left of=B]       {};
	\node[state]         (E) [below right of=B]       {};
	\node[state]         (F) [below left of=E]       {};

	\path (A) edge               node {} (B);
	\path (B) edge               node {} (C);
	\path (B) edge               node {} (D);
	\path (B) edge               node {} (E);
	\path (D) edge               node {} (F);
	\path (E) edge               node {} (F);
	\path (F) edge [out=0, in=0, distance=2cm] node [right] {} (B);
\end{tikzpicture}
\end{center}
\end{minipage}
\begin{minipage}[b]{0.5\linewidth}
\begin{center}
\begin{tikzpicture}[->,>=stealth',shorten >=1pt,auto,node distance=1.3cm,
semithick]
  \tikzstyle{every state}=[minimum size=0.4cm]
  \tikzstyle{PRstate}=[fill=black]

	\node[initial,state,PRstate] (A)                    {};
	\node[state,PRstate]         (B) [below of=A] {};
	\node[state,PRstate]         (C) [left of=B] {};

	\path (A) edge               node {} (B);
	\path (B) edge               node {} (C);
	\path (B) edge     [loop right]          node {} (B);
	\path (B) edge     [loop below]          node {} (B);
\end{tikzpicture}
\end{center}
\end{minipage}
\caption{En noir, les noeuds de $P$. À gauche, un graphe de
flot de contrôle, et à droite son équivalent avec uniquement les noeuds de $P$.
L'un des noeuds contient deux boucles, chacune correspondant à un chemin
possible dans le graphe de gauche.}
\end{figure}
Ce nouveau graphe contient moins d'états, mais un nombre potentiellement
exponentiellement plus grand que le graphe d'origine. C'est pourquoi la
technique consiste à ne pas construire explicitement ce graphe, et à procéder
comme suit:

Lors de l'analyse d'un noeud $i \in P$ du graphe, on encode la question suivante
sous forme de formule logique: ``Existe-t-il un chemin entre $i$ et $j \in P$,
tel que les variables numériques sont dans $A_i$ au noeud $i$, et tel que les
variables après le passage par ce chemin ne sont pas dans $A_j$ ?'', et on la
résout avec un SMT-solver. Si la réponse est ``oui'', celà veut dire qu'on a
trouvé un chemin qui fait avancer l'analyse, et on peut alors calculer la
transformation associée à ce chemin, et mettre à jour $A_j$. L'analyse d'un
noeud $i$ s'arrête lorsque plus aucun chemin n'est trouvé.

\section{Conclusion}

Nous proposons une méthode pour améliorer la précision de l'analyse par
Interprétation Abstraite, en guidant l'analyse vers les chemins du graphe à
explorer. Le fait de considérer des chemins complets et non pas de simples
transitions permet de gagner en précision, et l'utilisation du SMT-solving
permet de limiter le coût des calculs et la consommation mémoire.

\end{document}
