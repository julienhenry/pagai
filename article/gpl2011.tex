\documentclass{llncs}
\usepackage{a4wide}
\usepackage[french]{babel}
\usepackage[utf8]{inputenc}


\title{Analyse statique par Interprétation Abstraite}
\titlerunning{}
\subtitle{Raffinement de la méthode en utilisant le SMT-solving}
\author{Julien Henry}
\institute{CNRS - Verimag}

\begin{document}
\maketitle

\begin{abstract}
	L'analyse statique consiste à trouver à la compilation des propriétés sur un
	programme, en particulier l'ensemble des valeurs possibles pour chacunes des
	variables durant l'exécution. Ceci permet par exemple de prouver que le
	programme ne va pas faire de divisions par 0, ne va pas avoir de
	débordements arithmétiques, ou encore à calculer des invariants de boucle,
	etc. Trouver l'ensemble exact des valeurs possibles pour les variables est
	impossible, c'est pourquoi on a recours à des surapproximations : on calcule
	un ensemble plus grand, qui contient toutes les valeurs possibles des
	variables pendant l'exécution, mais aussi d'autres valeurs pour simplifier
	l'analyse : c'est le cas de l'Interprétation Abstraite, qui est une méthode
	bien connue de vérification statique.  
	On peut alors essayer de limiter cette surapproximation, grâce à une
	technique par SMT-solving, pour gagner en précision.
\end{abstract}

\section{Introduction}


\section{Interprétation abstraite}

\end{document}
